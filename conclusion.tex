\chapter{Conclusion} \label{conclusion}

The phenomenon of \textit{vanishing nodes} 
is investigated as another challenge when training deep networks.
Like the vanishing/exploding gradients problem, vanishing nodes also make training deep networks
difficult.
The hidden nodes in a deep neural network become more correlated as the network depth increases,
so the similarity between the hidden nodes increases.
Because similarity between nodes results in redundancy, the effective number of hidden nodes in a
network decreases.
This phenomenon is called\textit{"vanishing nodes"}.

To measure the degree of vanishing nodes, the \textit{Vanishing Nodes Indicator (VNI)} is proposed.
It is shown theoretically that the VNI is proportional to the network depth and inversely proportional
to the network width, which is consistent with the experimental results.
Via this theoretical tool,
we proof that the representation power of a network vanishes as the VNI goes to 1.
The effective number of nodes goes to 1 as when the VNI equals to 1, which is called
the "\textit{network collapsing}".
Also, we show that for a non-orthogonal initialized network, the VNI increases as the network
depth gets larger, and it asymptotically goes to 1 as the network depth tends to infinity.
That is, the network collapses when we consider a very deep feed-forward neural network.

However, if weight matrices are initialized with orthogonal distribution, or if a residual-like
architecture is applied, then the network will not collapse at a large depth.
We show theoretically that orthogonal weight can have small VNI at initial, and that
the network with identity shortcut connection is closer to the orthogonality.
Numerical simulations are also performed on different activation functions, weight initializations
and network architectures, which have a consistent result with our derivation.
Both theoretical and numerical results suggest that the weight initialization and the architecture
of a network determine its trainable depth.
Orthogonal weight initialization and residual-like architecture, from this point of view, are
relatively better for training a very deep neural network.

Moreover, we explore the difference between vanishing/exploding gradients and vanishing nodes,
and suggest a criterion to predict the occurrence of two problems by the network depth,
the network width, the activation, and the weight initialization. 
Finally, experimental results show that vanishing/exploding gradients and vanishing nodes are two
different challenges that make training deep neural networks difficult. 
%have demonstrated that the back-propagation training of networks intensifies the correlation between hidden nodes.
